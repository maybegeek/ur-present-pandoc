\documentclass[$if(lang)$$lang$,$endif$handout$if(aspectratio)$,aspectratio=$aspectratio$$endif$]{beamer}
%
\usepackage{fontspec,xunicode,xltxtra}
%
\defaultfontfeatures{Scale=MatchLowercase}
\setmainfont[Mapping=tex-text, Ligatures={Common, Rare, Historical}]{$if(mainfont)$$mainfont$$else$Frutiger Next LT W1G$endif$}
\setsansfont[Mapping=tex-text, Ligatures={Common, Rare, Historical}]{$if(sansfont)$$sansfont$$else$Frutiger Next LT W1G$endif$}
\setmonofont[Mapping=tex-text, Ligatures={Common, Rare, Historical}]{$if(monofont)$$monofont$$else$Frutiger Next LT W1G$endif$}
%
\usepackage{tikz}
\usetikzlibrary{positioning,shapes,shadows,arrows,decorations,decorations.pathmorphing,calc}
\usepackage{color,graphicx,listings,hyperref,amssymb,amsmath,fixltx2e,setspace}
%
\usepackage{beamerthemesplit}
\usetheme{/UR}

$if(natbib)$
\usepackage{natbib}
\bibliographystyle{plainnat}
$endif$
$if(biblatex)$
\usepackage{biblatex}
$if(biblio-files)$
\bibliography{$biblio-files$}
$endif$
$endif$
$if(listings)$
\usepackage{listings}
$endif$
$if(lhs)$
\lstnewenvironment{code}{\lstset{language=Haskell,basicstyle=\small\ttfamily}}{}
$endif$
$if(highlighting-macros)$
$highlighting-macros$
$endif$
$if(verbatim-in-note)$
\usepackage{fancyvrb}
$endif$
$if(tables)$
\usepackage{longtable}
% These lines are needed to make table captions work with longtable:
\makeatletter
\def\fnum@table{\tablename~\thetable}
\makeatother
$endif$
$if(url)$
\usepackage{url}
$endif$

$if(strikeout)$
\usepackage[normalem]{ulem}
% avoid problems with \sout in headers with hyperref:
\pdfstringdefDisableCommands{\renewcommand{\sout}{}}
$endif$
\setlength{\parindent}{0pt}
\setlength{\parskip}{6pt plus 2pt minus 1pt}
\setlength{\emergencystretch}{3em}  % prevent overfull lines
$if(numbersections)$
$else$
\setcounter{secnumdepth}{0}
$endif$
$if(verbatim-in-note)$
\VerbatimFootnotes % allows verbatim text in footnotes
$endif$
$if(lang)$
  \usepackage{polyglossia}
  \setmainlanguage{$mainlang$}
$endif$
$for(header-includes)$
$header-includes$
$endfor$

$if(title)$
\title[$shorttitle$]{$title$}
$endif$
$if(subtitle)$
\subtitle{$subtitle$}
$endif$
$if(institute)$
\institute[$shortinstitute$]{$institute$}
$endif$
$if(author)$
\author{$for(author)$$author$$sep$ \and $endfor$}
$endif$
$if(date)$
\date{$date$}
$endif$

\begin{document}
% Define the layers to draw the diagram
\pgfdeclarelayer{background}
\pgfdeclarelayer{foreground}
\pgfsetlayers{background,main,foreground}
\setbeamercovered{transparent}

{
  \usebackgroundtemplate{%
    \begin{tikzpicture}
      \coordinate (top) at (0,0);
      \coordinate (bottom) at (0,-.62\paperheight);
      \coordinate (left) at (-.51\paperwidth,0);
      \coordinate (right) at (.51\paperwidth,0);
      \node at ([xshift=.3\paperwidth,yshift=-.81\paperheight]left) %
      {\pgfuseimage{logo:text}};
      \fill [Neutralgrau] (bottom -| left) rectangle (top -| right);
      \fill [fakultaet] ([xshift=.344\paperwidth,yshift=-.15\paperheight]bottom -| left) rectangle (bottom -| right);
      % Now place the title page
      \begin{scope}[every node/.style={anchor=south west,inner sep=-8pt}]
      %\node[draw=green] at ([xshift=-.165\paperwidth]bottom) {.};
      \node at ([xshift=-.165\paperwidth]bottom) {%
        \begin{minipage}[bl]{.65\paperwidth}
          \setbeamertemplate{footline}{} 
          \titlepage
        \end{minipage}
      };
      \end{scope}
    \end{tikzpicture}
  }
  %\setbeamertemplate{headline}{\vspace{1cm}}
          \setbeamertemplate{footline}{} 
  \frame{}
\addtocounter{framenumber}{-1}
}

$for(include-before)$
$include-before$

$endfor$
$if(toc)$
\begin{frame}
\tableofcontents[hideallsubsections]
\end{frame}

$endif$
$body$

$if(natbib)$
$if(biblio-files)$
$if(biblio-title)$
$if(book-class)$
\renewcommand\bibname{$biblio-title$}
$else$
\renewcommand\refname{$biblio-title$}
$endif$
$endif$
\begin{frame}[allowframebreaks]{$biblio-title$}
\bibliography{$biblio-files$}
\end{frame}

$endif$
$endif$
$if(biblatex)$
\begin{frame}[allowframebreaks]{$biblio-title$}
\printbibliography[heading=none]
\end{frame}

$endif$
$for(include-after)$
$include-after$

$endfor$
\end{document}
