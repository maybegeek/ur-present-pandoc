\documentclass[ngerman,a4,footexclude,headinclude,DIV=9]{scrartcl}
\usepackage{lmodern}
\usepackage{amssymb,amsmath}
\usepackage{ifxetex,ifluatex}
\usepackage{fixltx2e}
\usepackage{mathspec}
\usepackage{xltxtra,xunicode}

\usepackage{fontspec}

\defaultfontfeatures{Mapping=tex-text,Scale=MatchLowercase}

\newcommand{\euro}{€}

    \setmainfont{Frutiger Next LT W1G}
    \setsansfont{Frutiger Next LT W1G}

% use upquote if available, for straight quotes in verbatim environments
\IfFileExists{upquote.sty}{\usepackage{upquote}}{}
% use microtype if available
\IfFileExists{microtype.sty}{%
\usepackage{microtype}
\UseMicrotypeSet[protrusion]{basicmath} % disable protrusion for tt fonts
}{}


\usepackage[bibstyle=%
fiwi2,%
citestyle=fiwi2,%
dashed=false,%
backend=biber,%
origyearwithyear=true,%
publisher=true,%
labeldate=true,%
pages=true,%
series=true,%
ibidtracker=false,%
filmruntime=true,%
citefilm=full%
]{biblatex}
\bibliography{Quellen/Quellen}


\usepackage{color}
\usepackage{fancyvrb}
\newcommand{\VerbBar}{|}
\newcommand{\VERB}{\Verb[commandchars=\\\{\}]}
\DefineVerbatimEnvironment{Highlighting}{Verbatim}{commandchars=\\\{\}}
% Add ',fontsize=\small' for more characters per line
\newenvironment{Shaded}{}{}
\newcommand{\KeywordTok}[1]{\textcolor[rgb]{0.00,0.44,0.13}{\textbf{{#1}}}}
\newcommand{\DataTypeTok}[1]{\textcolor[rgb]{0.56,0.13,0.00}{{#1}}}
\newcommand{\DecValTok}[1]{\textcolor[rgb]{0.25,0.63,0.44}{{#1}}}
\newcommand{\BaseNTok}[1]{\textcolor[rgb]{0.25,0.63,0.44}{{#1}}}
\newcommand{\FloatTok}[1]{\textcolor[rgb]{0.25,0.63,0.44}{{#1}}}
\newcommand{\CharTok}[1]{\textcolor[rgb]{0.25,0.44,0.63}{{#1}}}
\newcommand{\StringTok}[1]{\textcolor[rgb]{0.25,0.44,0.63}{{#1}}}
\newcommand{\CommentTok}[1]{\textcolor[rgb]{0.38,0.63,0.69}{\textit{{#1}}}}
\newcommand{\OtherTok}[1]{\textcolor[rgb]{0.00,0.44,0.13}{{#1}}}
\newcommand{\AlertTok}[1]{\textcolor[rgb]{1.00,0.00,0.00}{\textbf{{#1}}}}
\newcommand{\FunctionTok}[1]{\textcolor[rgb]{0.02,0.16,0.49}{{#1}}}
\newcommand{\RegionMarkerTok}[1]{{#1}}
\newcommand{\ErrorTok}[1]{\textcolor[rgb]{1.00,0.00,0.00}{\textbf{{#1}}}}
\newcommand{\NormalTok}[1]{{#1}}
\ifxetex
  \usepackage[setpagesize=false, % page size defined by xetex
              unicode=false, % unicode breaks when used with xetex
              xetex]{hyperref}
\else
  \usepackage[unicode=true]{hyperref}
\fi
\hypersetup{breaklinks=true,
            bookmarks=true,
            pdfauthor={Greiner \& Pfeiffer},
            pdftitle={Digitale Arbeitstechniken im Studium},
            colorlinks=true,
            citecolor=blue,
            urlcolor=blue,
            linkcolor=magenta,
            pdfborder={0 0 0}}
\urlstyle{same}  % don't use monospace font for urls
\setlength{\parindent}{0pt}
\setlength{\parskip}{6pt plus 2pt minus 1pt}
\setlength{\emergencystretch}{3em}  % prevent overfull lines
\setcounter{secnumdepth}{0}


\usepackage[ngerman]{babel}
\usepackage{polyglossia}
\setdefaultlanguage[spelling=new]{german}
%\usepackage[autostyle,german=guillemets]{csquotes}
\usepackage[autostyle,german=quotes]{csquotes}
\MakeOuterQuote{"}

\title{Digitale Arbeitstechniken im Studium}
\subtitle{gute wissenschaftliche Praxis}
\author{Greiner \& Pfeiffer}
\date{zigster Oktober}
\usepackage[absolute]{textpos}
%
\renewcommand*{\newunitpunct}{\adddot\space}
\renewcommand*{\titlepagestyle}{empty}
%
\makeatletter
\g@addto@macro\tableofcontents{\clearpage}
\g@addto@macro\listoftables{\thispagestyle{plain}}
\g@addto@macro\listoffigures{\thispagestyle{plain}}
\makeatother
%
\deffootnote[2em]{2em}{2.2em}{\thefootnotemark\ }
%
\usepackage{varioref}
% 
\renewcommand*{\sectionmarkformat}{}
%
\makeatletter
\newif \if@mainmatter \@mainmattertrue
\newcommand*\frontmatter{\clearpage\thispagestyle{plain}\@mainmatterfalse\pagenumbering{roman}}
\newcommand*\mainmatter{\clearpage\thispagestyle{plain}\@mainmattertrue\pagenumbering{arabic}}
\newcommand*\backmatter{\clearpage\thispagestyle{plain}\@mainmatterfalse}
\makeatother
%
%
\renewbibmacro*{filmtitle}
{\iffieldundef{maintitle}
{\printtext{\printfield[film]{title}}}
{\printfield[film]{maintitle}\newunit}%
\iffieldundef{subtitle}%
{}%
{\setunit{}%
\printtext{\addspace\printfield{subtitle}}}%
\iffieldundef{volume}%
{}%
{\printfield[season]{volume}}%
\iffieldundef{number}%
{}%
{\addcomma\addspace\printfield[episode]{number}}%
\iffieldundef{maintitle}%
{}%
{\addcolon\addspace\printfield[film]{title}}%
\addspace\mkbibparens{\printfield{year}\iffieldundef{origyear}{}{\printtext{/}\printorigdate}}%
\ifpunctmark{!}{\unspace .\newunit}{\addcolon}}%
%
%
\DeclareBibliographyDriver{movie}{%
  \usebibmacro{bibindex}%
  \usebibmacro{begentry}%
  \newblock%
  \usebibmacro{filmtitle}%
  \newunit\newblock%
  \usebibmacro{movie:creators}%
  \iffieldundef{entrysubtype}
	{}%
	{\iffieldequalstr{entrysubtype}{serial}%
		{\usebibmacro{movie:serials}}%
		{\iffieldequalstr{entrysubtype}{tv}%
			{\usebibmacro{movie:tv}}%
			{\usebibmacro{movie:regular}}}}%
  \iffieldundef{pagetotal}
  	{}
  	{%
 	\iftoggle{filmruntime}%
 		{\adddot\addspace\printfield{pagetotal}}%
 		{}}
 \iffieldundef{note}%
 	{}
 	{\printfield{note}}%
  \newunit\newblock
  \iftoggle{bbx:isbn}
    {\printfield{isan}}
    {}%
 \newunit\newblock
%\usebibmacro{doi+eprint+url}
 \usebibmacro{pageref}
 \iflistundef{location}%
 {}%
 {\printlist{location}\printtext{: }}
 \iffieldundef{howpublished}%
 {}%
 {\printfield{howpublished}}
 \newunit\newblock
 \usebibmacro{url+urldate}
\usebibmacro{finentry}}
%
%
\DeclareBibliographyDriver{book}{%
  \usebibmacro{bibindex}%
  \usebibmacro{begentry}%
  \iftoggle{dontprintorig}
  {}
  {\ifnameundef{author}%
  {\ifnameundef{editor}%
  {}
  {\usebibmacro{editor}\addspace}}%
  {\usebibmacro{author/translator+others}}%
  \usebibmacro{date+extrayear}}%
  \newblock
  \usebibmacro{mtitle+mstitle+vol+part+title+stitle}%
  \newunit\newblock
  \ifnameundef{author}
  	{}
	{\usebibmacro{byeditor+others}}%
  \newunit\newblock
  \printfield{note}%
  \newunit
  \printfield{volumes}%
  \newunit\newblock
  \usebibmacro{ser+num}%
  \newunit\newblock
  \printfield{edition}%
  \newunit\newblock%
  \usebibmacro{publ+loc+origyear}%
  \usebibmacro{chap+pag}%
  \newblock%\newunit
 \iffieldundef{howpublished}%
 {}%
 {\printfield{howpublished}\adddot}
  \usebibmacro{doi+eprint+url}%
  \addspace\usebibmacro{related}%
  \newunit\newblock
  \iftoggle{bbx:isbn}
    {\printfield{isbn}}
    {}%
  \newblock
  \usebibmacro{addendum+pubstate}%
  \newunit\newblock
  \usebibmacro{pageref}%
\finentry}
%
%
\renewbibmacro*{director:first-last}[4]{%
  \usebibmacro{name:delim}{#1#2#3}%
  \usebibmacro{name:hook}{#1#2#3}%
  \ifblank{#1}{}{\mkbibnamelast{#1}\isdot\addlowpenspace\addcomma\addspace}%
  \ifblank{#3}{}{%
    \mkbibnameprefix{#3}\isdot
    \ifpunctmark{'}
      {}
      {\ifuseprefix{\addhighpenspace}{\addlowpenspace}}}%
  \mkbibnamefirst{#2}\isdot
  \ifblank{#4}{}{\addlowpenspace\mkbibnameaffix{#4}\isdot}}
%
%
\renewbibmacro*{cite}{%
  \global\boolfalse{cbx:loccit}%
  \iffieldundef{shorthand}
    {\ifthenelse{\ifciteibid\AND\NOT\iffirstonpage}
       {\usebibmacro{cite:ibid}}
       {\ifthenelse{\ifnameundef{labelname}\OR\iffieldundef{labelyear}}
          %{\usebibmacro{cite:label}%
          {\printtext[bibhyperref]{\iffieldundef{shorttitle}{\printfield[film]{title}}{\printfield[film]{shorttitle}}}%
           \setunit{\addspace}}
          {\printnames{labelname}%
           \setunit{\nameyeardelim}}%
        \printfield{year}\iffieldundef{origyear}{}{\printtext{/}\printorigdate}}}%
    {\usebibmacro{cite:shorthand}}}    
%
%
\renewbibmacro*{textcite}{%
  \global\boolfalse{cbx:loccit}%
  \iffieldundef{type}%
  {%
  \ifnameundef{labelname}
    {\iffieldundef{shorthand}
       {\usebibmacro{cite:label}%
        \setunit{%
          \global\booltrue{cbx:parens}%
          \addspace\bibopenparen}%
        \ifnumequal{\value{citecount}}{1}
          {\usebibmacro{prenote}}
          {}%
        \usebibmacro{cite:labelyear+extrayear}}
       {\usebibmacro{cite:shorthand}}}
    {\printnames{labelname}%
     \setunit{%
       \global\booltrue{cbx:parens}%
       \addspace\bibopenparen}%
     \ifnumequal{\value{citecount}}{1}
       {\usebibmacro{prenote}}
       {}%
     \iffieldundef{shorthand}
       {\ifthenelse{\ifciteibid\AND\NOT\iffirstonpage}
          {\usebibmacro{cite:ibid}}
          {\iffieldundef{labelyear}
             {\usebibmacro{cite:label}}
             {\usebibmacro{cite:labelyear+extrayear}}}}
       {\usebibmacro{cite:shorthand}}}}%
       {\printtext[bibhyperref]{\iffieldundef{shorttitle}{\printfield[film]{title}}{\printfield[film]{shorttitle}}\addspace\mkbibparens{\printfield{year}}}}%
}
%
%
\renewbibmacro*{cite:label}{%
  \ifnameundef{labelname}
    {\BibliographyWarning{Missing author/editor+year or label}}
    {%
    \iffieldundef{type}{%
    \printtext[bibhyperref]{\printnames{labelname}}}%
    {\printfield{year}\iffieldundef{origyear}{}{\printtext{/}\printorigdate}}%
    }}
%
%
\renewcommand{\mkbibnamelast}[1]{\textsc{#1}}

\begin{document}

\maketitle

\setlength{\TPHorizModule}{1cm}

\setlength{\TPVertModule}{\TPHorizModule}

\begin{textblock}{4}(1.71,1.2)

\includegraphics[scale=0.9]{Template/TEX-PDF/logo-uni-rz-stud-it-regular}

\end{textblock}

{
\hypersetup{linkcolor=black}
\setcounter{tocdepth}{3}
\tableofcontents
\pagebreak
\clearpage
}
\section{Überschrift erster
Ordnung}\label{uxfcberschrift-erster-ordnung}

\emph{Markdown} formatierter Text. Hier kommt noch weiterer Text hin.

\section{nächste Folie}\label{nuxe4chste-folie}

\begin{itemize}
\itemsep1pt\parskip0pt\parsep0pt
\item
  Hier gibt es ein paar
\item
  Listenpunkte, welche ja
\item
  üblicherweise recht gern
\item
  verwendet werden.
\item
  \emph{Tufte} mag das nicht
\item
  ebensowenig \textcite[22-26]{dotzler:2008}
\end{itemize}

\textbf{Dieser Inhalt} erscheint nur in den "Speaker-Notes", bzw. im
Handout-Ausdruck.

Hier könnte man nun noch eine Referenz auf Tuftes wichtige Werke bzgl.
obiger Aussage machen. Oder auch nicht. Hier könnte man nun noch eine
Referenz auf Tuftes wichtige Werke bzgl. obiger Aussage machen. Oder
auch nicht. Hier könnte man nun noch eine Referenz auf Tuftes wichtige
Werke bzgl. obiger Aussage machen. Oder auch nicht.

\section{alles in runden Klammern}\label{alles-in-runden-klammern}

\begin{itemize}
\itemsep1pt\parskip0pt\parsep0pt
\item
  \autocite{dotzler:2008}\\ \texttt{{[}@dotzler:2008{]}}
\item
  \autocite[9-12]{dotzler:2008}\\ \texttt{{[}@dotzler:2008, 9-12{]}}
\item
  \autocites[9-12]{dotzler:2008}[außerdem][7-13]{dotzler:1995}\\
  \texttt{{[}@dotzler:2008, 9-12; außerdem @dotzler:1995, 7-13{]}}
\end{itemize}

\section{Unterdrückung des Autors im
Klammernausdruck.}\label{unterdruxfcckung-des-autors-im-klammernausdruck.}

\begin{itemize}
\itemsep1pt\parskip0pt\parsep0pt
\item
  \autocite*{dotzler:2008}\\ \texttt{{[}-@dotzler:2008{]}}
\item
  \autocite*[9-12]{dotzler:2008}\\ \texttt{{[}-@dotzler:2008, 9-12{]}}
\item
  \autocites[9-12]{dotzler:2008}[außerdem][7-13]{dotzler:1995}\\
  \texttt{{[}-@dotzler:2008, 9-12; außerdem -@dotzler:1995, 7-13{]}}
\end{itemize}

\section{Autor vor die Klammern
ziehen}\label{autor-vor-die-klammern-ziehen}

\begin{itemize}
\itemsep1pt\parskip0pt\parsep0pt
\item
  \textcite{dotzler:2008}\\ \texttt{@dotzler:2008}
\item
  \textcite[9-12]{dotzler:2008}\\ \texttt{@dotzler:2008 {[}9-12{]}}
\item
  \textcite[9-12]{dotzler:2008} außerdem \textcite[7-13]{dotzler:1995}\\
  \texttt{@dotzler:2008 {[}9-12{]}; außerdem @dotzler:1995 {[}7-13{]}}
\end{itemize}

\section{weitere Folie}\label{weitere-folie}

\begin{quote}
Auch hier gibt es wieder etwas Text. Auch hier gibt es wieder etwas
Text. Auch hier gibt es wieder etwas Text. Wohl ein langes Zitat.
\end{quote}

\section{diese Folie}\label{diese-folie}

Also diese Folie beinhaltet nun ein paar Auszeichnungen, für eine
\emph{leichte Hervorhebung} und für eine \textbf{starke Hervorhebung}.

Ebenfalls wichtig ist die Möglichkeit der Verlinkung. Als Beispiel auf
die \href{http://www.ur.de}{Website der Universität Regensburg}.

\section{noch eine letzte Folie}\label{noch-eine-letzte-folie}

Quelltext-Beispiel

\begin{Shaded}
\begin{Highlighting}[]
\DataTypeTok{<!DOCTYPE }\NormalTok{html}\DataTypeTok{>}
\KeywordTok{<html>}
    \KeywordTok{<head>}
        \KeywordTok{<meta}\OtherTok{ charset=}\StringTok{"utf-8"}\KeywordTok{>}
        \KeywordTok{<title>}\NormalTok{SITA}\KeywordTok{</title>}
    \KeywordTok{</head>}
    \KeywordTok{<body>}
        \KeywordTok{<h1>}\NormalTok{Hallo SITA}\KeywordTok{</h1>}
    \KeywordTok{</body>}
\KeywordTok{</html>}
\end{Highlighting}
\end{Shaded}

\section{last but not least}\label{last-but-not-least}

Vielen Dank für Ihre Aufmerksamkeit.



\clearpage
\thispagestyle{plain}
\markboth{}{}


\printbibheading[heading=bibintoc,title={Quellen}]
\printbibliography[notkeyword=film,title={Literaturverzeichnis}]
\printbibliography[keyword=film,title={Filmverzeichnis}]



\end{document}
